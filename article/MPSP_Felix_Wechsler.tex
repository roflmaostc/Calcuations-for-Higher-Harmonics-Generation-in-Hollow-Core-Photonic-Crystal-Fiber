\documentclass[fleqn, 10pt, twocolumn]{SelfArx}
\usepackage[english]{babel} 
\usepackage[utf8]{inputenc}



\setlength{\columnsep}{0.55cm} % Distance between the two columns of text
\setlength{\fboxrule}{0.75pt} % Width of the border around the abstract

\usepackage[version=4]{mhchem}

\usepackage{fancyhdr}
\usepackage{here}
\usepackage{relsize}

\usepackage[locale=DE,output-decimal-marker={.}]{siunitx}
\usepackage{amsmath}
\sisetup{separate-uncertainty, per-mode=fraction,}
\usepackage{tikz}
\usepackage{pgfplots,pgfplotstable}
\usetikzlibrary{external}
\tikzexternalize[prefix=tikz/]

\usepackage{hyperref}


\pagestyle{fancy}
\fancyhf{}
\lhead{Felix Wechsler}
\rhead{Topic 3 \textit{Photonic crystal fibers}}
\definecolor{color1}{RGB}{0,0,90} % Color of the article title and sections
\definecolor{color2}{RGB}{0,20,20} % Color of the boxes behind the abstract and headings


\usepackage[%
  backend=biber,
  url=false,
  %style=alphabetic,
  % citestyle=authoryear,
  citestyle=numeric,
  maxnames=1,
  minnames=1,
  maxbibnames=99,
  giveninits,
  uniquename=init]{biblatex}
\DeclareSourcemap{
  \maps[datatype=bibtex]{
    \map{
      \step[fieldset=issn, null]
    }
  }
}
\DeclareSourcemap{
  \maps[datatype=bibtex]{
    \map{
      \step[fieldset=doi, null]
    }
  }
}
\DeclareSourcemap{
  \maps[datatype=bibtex]{
    \map{
      \step[fieldset=isbn, null]
    }
  }
}


\addbibresource{references.bib}
\usepackage[symbol]{footmisc}
\renewcommand{\thefootnote}{\fnsymbol{footnote}}

\begin{document}%
\begin{center}
\colorbox{color2!10}{\large\relscale{1.13}\textcolor{color1}{\sffamily\bfseries Entrance Exam for MPSP Applicants}}
\end{center}
    \textit{Third harmonics can be generated by pumping Ti:sapphire laser femtosecond pulses into an Ar-filled hollow-core photonics crystal fiber (HC-PCF) \cite{Nold2010}. 
        In this work\footnote{For more details about the numerical calculations see this GitHub repository \url{https://git.io/Jt95W}} we discuss several aspects of their generation and analysis.}

    \section{Derivation of Effective Refractive Index}
    The effective refractive index of the generated modes within the HC-PCF can by approximated by:
    \begin{align}
        n^{mn}(\lambda, p) = \sqrt{n^2_{\text{Ar}}(\lambda, p ) - \frac{u_{mn}^2}{k_0^2 a^2}} 
        \label{eq1}
    \end{align}
    where $k_0=2\pi/\lambda$ is the wave number, $\lambda$ is the vacuum wavelength, $p_0$ and $p$ are a reference and the present pressure, respectively. $a$ is the 
    effective radius of the fiber. $u_{mn}$ is the $n$th zero of a $m$th order Bessel function of the first kind. Since the fiber used in \cite{Nold2010} has a hexagonal core, \eqref{eq1} is only an approximation.
    \eqref{eq1} originates from two principles: a $\lambda$ dependent refractive index and an effective refractive index due to the present mode in a circular fiber.
    Firstly, we discuss the overall structure of the equation coming from the mode formed within the fiber. The field in the fiber is generated by a laser pulse. Once the laser is launched, Maxwell's equation and hence the Helmholtz equation have to be valid.
    % \begin{align}
    %     \left(\nabla^2 + k(\omega)\right) \mathbf E = 0
    %     \label{eq:helmholtz}
    % \end{align}
    It is known that the Helmholtz equation within a fiber with circular cross section is solved by Bessel functions \cite{Nickelson2019}. 
    Numerically, a Finite Difference Mode Solver can be used to obtain the eigenvectors $\mathbf E$ representing the field and the eigenvalues $k^{mn}$ representing the wave number for more complex shapes. The analytical solution for cylindrical symmetries is 
    \begin{align}
        k^{mn} = \sqrt{(n k_0)^2 - \left( \frac{u_{mn}}{a}\right)^2}
    \end{align}
    where $n$ is the refractive index of the medium.
    From the wave number we obtain the effective refractive index 
    \begin{align}
        n^{mn} = \sqrt{n^2 - \left( \frac{u_{mn}}{a k_0}\right)^2}.
        \label{eq:mode}
    \end{align}
    Secondly, we can now introduce a wavelength and a pressure dependency to the equation. To describe the $\lambda$ dependency, several empirical approaches are possible. 
    In the case of noble gases, the following Sellmeier equation turned out to be convenient \cite{Borzsonyi2008}
    \begin{align}
       n^2(\lambda, p_0)  = 1 + \frac{T_0}{T} \left(\frac{B_1}{1- (\lambda_1/\lambda)^2} + \frac{B_2}{1- (\lambda_2/\lambda)^2}  \right)
        \label{eq:wvl}
    \end{align}
    where $B_{1/2}$ and $\lambda_{1/2}$ are fitting parameters being gas specific.
    Furthermore, from the Lorentz-Lorenz theory
    we know that there is a linear relation between the squared refractive index and the pressure $p$ \cite{Borzsonyi2008}.
    Knowing $n^2(p_0) \sim p_0$ it is straightforward to see that
    \begin{align}
        n_{\text{Ar}}^2(\lambda, p) = \frac{p}{p_0} n^2(\lambda, p_0)
        \label{eq:p}
    \end{align}
    and therefore by measuring the Sellmeier parameters for a pressure $p_0$ we can plug Equation \ref{eq:wvl}, \ref{eq:p} into \ref{eq:mode} to obtain \eqref{eq1}. 
    By introducing a first order Taylor approximation of $\sqrt{1+x} \approx 1 + \frac{x}{2}$, we can further simplify to obtain the equation below.
    \begin{align}
        n^{mn}(\lambda, p) \approx 1 + n^2(\lambda, p_0) \frac{p}{2p_0} -  \frac{u_{mn}^2}{2 k_0^2 a^2}
        \label{eq1a}
    \end{align}

    \section{Phase Matching}
    In third harmonics generation, two fundamental conditions need to hold: energy and momentum conservation. 
    In the wave picture, momentum conservation is a phase matching condition and can be in the fiber mode expressed as
    \begin{align}
        \mathbf{k}^{mn} = 3 \cdot \mathbf k^{11} 
    \end{align}
    and hence
    \begin{align}
        n^{11}(\lambda, p) - n^{mn}(\lambda / 3, p)= 0
        \label{eq:pm}
    \end{align}
    where $mn$ is the generated third harmonic mode and $11$ the pump mode. 
    The left hand side of \eqref{eq:pm} can be visualized to find all possible solutions. 
    We believe\footnote{Despite \citeauthor{Travers2011} state differently \cite{Travers2011}, we could only reproduce the results of \citeauthor{Nold2010} by decrementing $m$. This decrementation is mentioned by \citeauthor{Marcatili1964} \cite{Marcatili1964}.} that the hybrid mode $\text{HE}_{mn}$ corresponds to Bessel coefficients $u_{(m-1)n}$. 
    Accordingly, $\text{HE}_{11}$ has the Bessel coefficients $u_{01}$.
    \begin{figure}[h]
        \centering
        \begin{tikzpicture}
            \begin{axis}[
                xmin = 400, xmax = 1000,
                ymin = -5, ymax = 7.5,
                ytick = {-2.5, 0, ..., 7.5},
                height=5.5cm, width=8cm,
                xlabel = {$\lambda$ in \si{\nano\meter}}, 
                ylabel = {$(n^{11}(\lambda, p) - n^{mn}(\lambda/3, p)) \cdot 10^4$},
                legend pos = {north west},
                legend entries = {$\ce{Ar}-\text{HE}_{11}$, $\ce{Ar}-\text{HE}_{12}$, $\ce{Ar}-\text{HE}_{13}$, $\ce{Ar}-\text{HE}_{23}$, $\ce{Ar}-\text{HE}_{33}$}
                ]
                \draw[dashed] (axis cs:0,0) -- (axis cs:1200, 0);
                \draw[dashed] (axis cs:800,10) -- (axis cs:800, -10);
                \node at (axis cs:860, 5) {\footnotesize pump $\lambda$};
                \addplot[color=blue]
                    table[x index = 0, y index = 1]{../data/argon.txt};
                \addplot[color=red]
                    table[x index = 0, y index = 2]{../data/argon.txt};
                \addplot[color=green]
                    table[x index = 0, y index = 3]{../data/argon.txt};
                \addplot[color=orange]
                    table[x index = 0, y index = 4]{../data/argon.txt};
                \addplot[color=black!50!white]
                    table[x index = 0, y index = 5]{../data/argon.txt};
                \addplot[color=blue, dotted]
                    table[x index = 0, y index = 1]{../data/argon_low_pressure.txt};
            \end{axis}
        \end{tikzpicture}
        \caption{Refractive index difference of \ce{Ar} for different \text{HE} modes and $\text{HE}_{11}$ at $T=\SI{293}{\kelvin}$, $p=\SI{5000}{\milli\bar}$.}
        \label{plt:pm}
    \end{figure}
    From \autoref{plt:pm} it follows that intra-modal phase matching ($mn = 11$) is not possible since the $\text{HE}_{11}$ curve does not reach values where the refractive index difference is approximately 0.
    Even, at a low pressure of $p=\SI{100}{\milli\bar}$ the dotted blue line does not reach 0. 
    Also mathematically it is clear that \eqref{eq:pm} cannot be solved for intra-modal phase matching because the only variable is $\lambda$ and since $n(\lambda)$ is a monotonic decreasing function, there is no solution for $\lambda$ and $\lambda/3$. 
    However, $\text{HE}_{13}$ has a solution to the phase matching condition identifiable at $\lambda=\SI{800}{\nano\meter}$.
    Mathematically, $\lambda/3$ leads to a decrease in \eqref{eq:pm} which is then compensated by an increase of $u_{03}>u_{01}$.\\
    Instead of using \ce{Ar}-filled HC-PCF, third harmonics can be generated with \ce{Kr} as well. 
    Using the Sellmeier coefficients for \ce{Kr}, we apply the phase matching condition to 
    solve for the required pressure.
    \begin{figure}[h]
        \centering
        \begin{tikzpicture}
            \begin{axis}[
                xmin = 0, xmax = 9,
                ymin = -3, ymax = 5.5,
                ytick = {-2, 0, 2, 4, 6},
                height=5.5cm, width=8cm,
                xlabel = {$p$ in \si{\bar}}, 
                ylabel = {$(n^{11}(\lambda, p) - n^{mn}(\lambda/3, p)) \cdot 10^4$ },
                legend pos = {north west},
                legend entries = {$\ce{Kr}-\text{HE}_{13}$, $\ce{Kr}-\text{HE}_{23}$, $\ce{Kr}-\text{HE}_{33}$}
                ]
                \draw[dashed] (axis cs:0,0) -- (axis cs:12, 0);
                \addplot[color=blue]
                    table[x index = 0, y index = 1]{../data/krypton.txt};
                \addplot[color=red]
                    table[x index = 0, y index = 2]{../data/krypton.txt};
                \addplot[color=green]
                    table[x index = 0, y index = 3]{../data/krypton.txt};
            \end{axis}
        \end{tikzpicture}
        \caption{Refractive index difference of \ce{Kr} for different \text{HE} modes and $\text{HE}_{11}$ at $T=\SI{293}{\kelvin}$,  $\lambda=\SI{800}{\nano\meter}$.}
        \label{plt:pressure}
    \end{figure}
    In \autoref{plt:pressure} we can see that for $p^{13}=\SI{2.4}{\bar}$, $p^{23}=\SI{5.3}{\bar}$ and $p^{33}=\SI{8.5}{\bar}$ the phase matching is fulfilled. 
\section{\ce{Xe}-filled HC-PCF}
    \autoref{plt:pmxenon} shows the refractive index difference for a \ce{Xe}-filled HC-PCF\footnote{Note, there is a typo in \cite{Borzsonyi2008}. $C_1$ should be \SI{12.75e-3}{\micro\meter\squared} instead of \SI{12.75e-6}{\micro\meter\squared}. See \url{https://refractiveindex.info/?shelf=main&book=Xe&page=Borzsonyi}}.
    \begin{figure}[h]
        \centering
        \begin{tikzpicture}
            \begin{axis}[
                xmin = 400, xmax = 1000,
                ymin = -5, ymax = 7.5,
                ytick = {-2.5, 0, ..., 7.5},
                height=5.5cm, width=8cm,
                xlabel = {$\lambda$ in \si{\nano\meter}}, 
                ylabel = {$(n^{11}(\lambda, p) - n^{mn}(\lambda/3, p)) \cdot 10^4$},
                legend pos = {north west},
                legend entries = {$\ce{Xe}-\text{HE}_{11}$, $\ce{Xe}-\text{HE}_{12}$,$\ce{Xe}-\text{HE}_{13}$, $\ce{Xe}-\text{HE}_{23}$, $\ce{Xe}-\text{HE}_{33}$}
                ]
                \draw[dashed] (axis cs:0,0) -- (axis cs:1200, 0);
                \draw[dashed] (axis cs:800,10) -- (axis cs:800, -10);
                \node at (axis cs:860, 5) {\footnotesize pump $\lambda$};
                \addplot[color=blue]
                    table[x index = 0, y index = 1]{../data/xenon.txt};
                \addplot[color=red]
                    table[x index = 0, y index = 2]{../data/xenon.txt};
                \addplot[color=green]
                    table[x index = 0, y index = 3]{../data/xenon.txt};
                \addplot[color=orange]
                    table[x index = 0, y index = 4]{../data/xenon.txt};
                \addplot[color=black!50!white]
                    table[x index = 0, y index = 5]{../data/xenon.txt};
            \end{axis}
        \end{tikzpicture}
        \caption{Refractive index difference of \ce{Xe} for different \text{HE} modes and $\text{HE}_{11}$ at $T=\SI{293}{\kelvin}$, $p=\SI{960}{\milli\bar}$.}
        \label{plt:pmxenon}
    \end{figure}
    It follows that third harmonic generation is possible for \ce{Xe} at lower pressures but even higher harmonics have been observed in \ce{Xe}
    \cite{Heckl2009}. Among the investigated noble gases, \ce{Xe} has the lowest absolute pressure and \ce{Ar} the highest required for third harmonic generation. 
    The reason for the lower pressure is that \ce{Xe} has the largest dispersion \cite{Borzsonyi2008} leading to a smaller $p$ needed to compensate for the larger $u_{mn}$ in \eqref{eq:pm}.
    Consequently \ce{Ar} shows the lowest dispersion.
    It is worth to mention that the opportunity to use different gases enhances the use in practical applications of higher harmonics generation with HC-PCFs. 
    \ce{Xe} and \ce{Ar} allow to vary the presssure by almost one order of magnitude but still generating third harmonics. Also, the linear pressure dependency in the phase matching condition
    allows to fine-tune efficiently for certain higher order modes.
    Furthermore, combining \eqref{eq1a} together with the ideal gas law, gas-filled HC-PCFs are robust against temperature induced pressure variations. For moderate gas chamber dimensions temperature changes due to absorption of radiation are negligible \cite{Serebryannikov2004}.
    In conclusion, noble gas-filled HC-PCFs allow for setups to be small, nonhazardous and insensitive to environmental changes and are therefore 
    suitable candidates for generation of light in UV or EUV regimes.
    It can be conjectured, that other noble gases like Helium and Neon are also promising candidates since their Sellmeier coefficients are in the same order of magnitudes and the physical properties of noble gases are similar. 

    \printbibliography

\end{document}
